% !TEX root = ../rapport.tex
\newpage

\section{Motor omdrejninger (kode)}
Dette kodemodulet er lavet til at måle tiden mellem pulsene fra motor encoderen. Tiden gemmes i sram, hvorved andre kodemoduler kan bruge værdien. Ved både at kende tiden imellem pulsene og antal af pulse på en motor omgang, kan motorens omdrejnings hastigheden beregnes. Til at måle tiden, benyttes en macro fra afsnit \ref{lapt}. Ud over at måle tiden imellem hvert puls, så bliver antallet af impulser fra motor encoderen også talt. Pulsene fra motor encoder vil altid repræsentere samme fysiske distance, uafhængigt af bilens hastighed. Antallet af pulser vil derfor repræsentere hvor stor en strækning bilen har tilbagelagt.  

\subsection{Kode}
Koden består af en interrupt rutine, som bliver kørt når "Ekstern interrupt 1" trigges. Motor encoderen er tilsluttet det eksterne interrupt, så hver gang der kommer et puls fra encoderen, bliver rutinen kørt. Figur \ref{fig:physs} fra bilaget, viser et flowchart over koden.\\
\\
Først pushes værdierne fra general purpose registrene, samt status registret ind i stakpointeren. Dernæst loades den nuværende timer værdi. Timer værdien trækkes fra en "gammel" timer værdi, fra forrige encoder impuls, for at finde forskellen mellem de to timer værdier. Forskellen mellem de 2 værdier, repræsentere den tid der er gået mellem de 2 pulse. Tiden mellem de 2 pulse ligges ind i SRAM, så andre kodemoduler let kan benytte værdien. Herefter inkrementeres distance registret, for at holde styr på hvor langt bilen har kørt. For at have den rigtige timer reference værdi, næste gang denne rutine kaldes, overskrives den "gmale" timer værdi med den "nuværende" timer værdi. Inden der returneres fra rutinen, popes de oprindelige værdier for henholdsvis general purpus registrene og status registret.

\subsection{Macro's}
For at gøre de 2 værdier lettilgængeligt for andre kodemoduler, er der lavet 2 macro'er. Den første makro returnere tiden mellem pulsene og den anden macro returnere den tilbagelagte distance i for af et antallet af encoder pulser, siden start linjen blev passeret.