% !TEX root = ../rapport.tex

\section{Fysiske modifikationer og hardware}
Som tidligere nævnt skal bilen både optimeres fysisk og ved hjælp af software.
I dette kapitel bliver der sat fokus på hvordan bilen er blevet optimeret fysisk, og hvilket hardware der er blevet udviklet til bilen.
For at optimere bilens omgangstid mest muligt, er der også fortaget fysiske modifikationer på bilen.
Som beskrevet i opgaveformulering var der krav til at der skulle bruges enten en elektromagnet som aktuator, eller en hall-sensor.
Der er blevet foretrukket en elektromagnet til at skabe yderligere downforce.
De ting der blandt andet er med til at give en optimeret omgangstid er:
\begin{itemize}
\item Minimering af vægt.
\item Justerbar downforce.
\item Lavere tyngdepunkt.
\item Afstivning af chassis.
\item Minimering af bremselængde.
\item Maksimering af acceleration og tophastighed.
\end{itemize}


% !TEX root = ../rapport.tex

\subsection{Fysiske modifikationer}

For at minimere vægten af bilen er alt unødvendigt plastik blevet fjernet fra bilen.

Downforce er ekstremt brugbart i sving da det hjælper med at holde bilen på banen ved højere fart. Downforce er mindre brugbart på lige strækninger da bilen let bliver på banen og downforce faktisk vil sænke bilens acceleration og topfart. Derfor indføres der justerbar downforce med elektromagneten. I det 3D-printet chassis er der en indbygget holder til elektromagneten. Elektromagnet er placeret bagerst for at give kontra til den permamagnet der i forvejen sidder i bilen. Permamagneten er placeret foran og tæt på splitten for at trække splitten ned i banen. For uddybning af elektromagneten se kapitel [noget].

\subsubsection{3D-printet chassis}

For at kunne hurtigt gennem sving kræver det et lavt tyngdepunkt i bilen, jo lavere tyngdepunktet er jo hurtigere kan bilen klare at køre gennem et sving uden at kæntre. Ved at fjerne affjedringen optimeres tyngdepunktet også da karrosseriet er i sin laveste tilstand hele tiden, og ikke kun når fjedrene er trykket sammen.

Det udleveret chassis på racerbilen er meget vakkelvornt, og kan vrides let. For at optimere dette er der blevet 3D-printet et nyt forstærket chassis. Det chassis er stærkere end det gamle og giver lettere plads til at montere sensorer og aktuatore i bilen. Det nye chassis er tegnet i AutoDesk Inventor og derefter 3D-printet med Makerbot. 3D-printet er printet i flere dele og derefter samlet med skruer. En fordel ved at printet er i flere dele er hvis én del går i stykker skal man ikke printe det hele igen. 3D-tegningen ses på figur \ref{fig:3Dtegning} nedenfor.

\begin{figure}[ht]
    \centering
    \includegraphics[width=0.7\textwidth]{kapitler/billeder/3Dtegning.jpg}
    \caption{3D-tegning af bilens chassis}
    \label{fig:3Dtegning}
\end{figure}


Det vil sige at den affjedring der følger med bilen er blevet fjernet og erstattet med det afstivet chassis. Grunden til man har affjedring i virkelige biler er for at beskytte både kører og motor, da man aldrig er sikker på asfaltens tilstand. Da banen racerbilen skal køre på er flad og der er ingen kører der skal beskyttes, vil det optimere bilens hastighed igennem sving uden at den kæntre, at fjerne affjedringen. På figur \ref{fig:racerchassis} nedenfor ses det 3D-printet chassis med motor og hjul på.

\begin{figure}[ht]
    \centering
    \includegraphics[width=0.7\textwidth]{kapitler/billeder/racerchassis.jpg}
    \caption{3D-printet chassis med motor og hjulakser}
    \label{fig:racerchassis}
\end{figure}

Det nye 3D-printet chassis er også udviklet til at ATmega-boardet kan skrues direkte på. Det nye chassis er blevet testet mod det gamle chassis. Strømmen fra banen bliver sat direkte til motoren så hastigheden kan justeres ved at regulere spændingen. Med det gamle chassis røg bilen af banen ved ca. 7V i et 90 grader sving, men med det nye chassis kan det kører gennem et 90 grader sving med 10V. Derved kan det ses at det nye chassis opnår det ønsket formål. 




% !TEX root = ../rapport.tex

\subsection{Elektromagnet}
\label{Elektromagnet}
?Fysisk analyse af bilen i sving, ide princip. Bilen drejer hvilke krafter påvirker denne. Kraften på mængden øverst i bilen større end kraften på ting tætte ved banen. Tegning af bil kraft moment osv. Enten præcis antal Newton eller så mange newton tilgængelig.?

Elektromagneten er først og fremmest begrænset af bilens dimensioner. I vores tilfælde har vi lavet et fri rum på bredden 1.90cm x længden 1.73cm x højden 1.37cm = 4,50 cm3. 
Det betyder at elektromagnetens kerne med spole max må bliver 1.2 cm høj, 1.5cm bred (1.5  bred banen 1.1 ) og 1.5 cm lang ( volumen 2.7 cm3). (Emagnet højde + ben = 1.4 cm)

Med en kerne med et volumen på 0.3 $cm^3$ (Indskrevne cirkel af en firkant på $0.3cm \cdot 0.3 cm$ ) giver det os resterne 2.4 $cm^3$ til spolen.
Det giver os et estimat for hvor mange vindinger N, vi kan havde med en ledning med tværsnitsareal Aw, med en spole med gennemsnit radius på r som er 0.45 cm. (*Gennemsnit omkreds da omkredsen stiger linært i takt med der kommer viklinger. 0.3cm for kernen, spole slutter ved 0.6cm det giver gns på 0.45 cm….. *Givet at viklingerne ligger helt tæt. )


\begin{equation}
Vol_{max} =\pi \cdot r \cdot 2 \cdot A_w \cdot N 
\end{equation}
\begin{equation}
N= \frac{Vol_{max}}{r \cdot pi \cdot 2 \cdot aw}
\end{equation}

Det er vigtigt at vælge en ledning med et $A_{w}$ der kan holde til at man sender den mængde ampere igennem den uden at lakeringen begynder at smelte, eller det omkring liggende PLA chassis bliver blødt (Bilens steel). 
Hvis man kigger på anbefalingerne for hvad max strømmen i en kobber ledning og dens isolation er, kan man se for 1 A anbefaledes ledning med en diameter på 0.28 mm. Det er dog en anbefaling for længere vedvarende brug, hvor vi kun skal bruge vores i små korte perioder.

$http://www.powerstream.com/Wire_Size.htm $

Vi fandt at vi kunne bruge en 0.20 mm kobbertråd, uden at lakering smeltede eller PLA chassis blev blødt. Hvor hvis vi gik ned i størrelse begyndte magneten at blive for varm til vi var trygge ved at havde den i bilen.
Teoretisk skulle vi så kunne lave en magnet med følgende antal vindinger:

\begin{equation}
N=\frac{Vol_{max}}{r \cdot pi \cdot 2 \cdot aw}  ?indsæt værdier?? = x
\end{equation}


Men da vi ikke kan ligge vindingerne helt tætte og ikke har en helt rund kerne har det i praksis kun været muligt for os at lave 600 vindinger.
Vi brugte en ”spole maskine” – tjek hvad den hedder med Jan. For at lave vindingerne. Vi fandt dog ud af at alene på de 100 viklinger kunne man lave 1.5 gange flere viklinger ved at gøre det i hånden, dog tog det også væsentlig længere tid.
Den kraft elektromagneten vil kunne trække bilen ned med kan estimeres ved først at kigge på den magnetiske energi der vil blive opladt i DeltaV matriale med  en ”konstant”?er stål  og evt jern dette?? permabiliteten my. 

\begin{equation}
W_m =\frac{B^2}{2 \mu} \Delta V
\end{equation}


Hvis vi integrere energi densiteten over volumen af hele det magnetiske kredsløb og antager at B-feltet bliver holdt konstant igennem tværsnits arealet A på elektromagneten, medføre det at:
\begin{equation}
\Phi=B*A
\end{equation}
\begin{equation}
B=\frac{\Phi}{A} 
\end{equation}


I praktisk vil dette ikke være helt sandt da der vil opstå nogle hvirvel strømme mm. TJEK OP!! Hvilke. Husk vi antager at $\mu_0$ konstant, og Phi konstant.  Men i vores tilfælde kan vi forstille os at fluxen løber igennem kredsløbet som vand ville i et rør.

\begin{equation}
W_m = \frac{1}{2} \int B^2 \frac{dv}{\mu}  = \frac{1}{2} \Phi^2 \int \frac{1}{A^2 \cdot \mu} dV
\end{equation}

Det gælder for et magnetisk kredsløb at den resulterende flux $\Phi$ er givet ved:

\begin{equation}
\Phi=\mu H \cdot A  \semequ \frac{NI}{R_m} \semequ  \frac{NI}{\frac{1}{A\mu_0}}
\end{equation}

$N \cdot I$ er ampere vindinger, Rm er reluktansen, $ \mu $ er permeabiliteten og l længden af kredsløbet. Det medføre at vi kan skrive energien som.
\begin{equation}
W_m = (NI)^2/(Rm)^2 Integrer (dl/(A \cdot \mu_0))
\end{equation}


Energi balancen i vores kredsløb er.
$mekanisk arbejde+ magnetisk energi(Tjek opladt energi)=Elektrisk energi \cdot tid$

\begin{equation}
F dx + dW_m = iv dt
\end{equation}


Givet at vi holder en konstant strøm i spolen vil den elektrisk energi skrive som.

\begin{equation}
iv=IN d\Phi /dt=(NI)^2  d(1/R_m )/dt
\end{equation}
  

Sammensat ligning x og y får vi

\begin{equation}
F dx+(NI)^2/2  d (1/R_m )  =(NI)^2  d(1/R_m )
\end{equation}

Sidste del tjekkes med jan

PWM for at holde konstant strøm uddybet 

Sammenligning magneter

Sammenligninger kørsel med og uden magnet
% !TEX root = ../rapport.tex

\subsection{Bremsefunktion (H-bro)}

Ligesom acceleration, er bremselængden en vigtig faktor for at opnå en god omgangstid.
Når bilen nærmer sig et sving, handler det om at komme så tæt på svinget som muligt, før at bilen
skal begynde at bremse. Dette kræver en effektiv bremse, så bremselængden minimeres og omgangstiden effektiviseres.
For at bremse bilen så effektivt som muligt, benyttes princippet bag en H-bro. Dette princip er illustreret
på figur \ref{fig:hbro}, nedenfor.

\begin{figure}[ht]
    \centering
    \includegraphics[width=1\textwidth]{kapitler/billeder/hbro.png}
    \caption{Viser princippet bag en H-bro, og hvordan den benyttes i dette projekt.}
    \label{fig:hbro}
\end{figure}

Som illustreret på figur \ref{fig:hbro}, kan man bruge en H-bro til at vende strømmen igennem motoren.
Ved at vende strømmen igennem motoren i få millisekunder, vil motoren prøve at bakke,
og bilen vil bremse meget effektivt. Figuren viser det to scenarier, hvor strømmen til højre er vendt
og strømmen til venstre ikke er vendt.

H-broen konstrueres med et dobbelt relæ, som vender de to kontakter når spolen inducere en strøm.
Spolen kræver en høj aktiveres spænding for at trække den nødvendige strøm
og kredsløbet bygges derfor op med en transistor.
Transistoren kan aktiveres med en meget lille strøm og spænding, og kan derfor aktiveres
af ATmega'ens I/O pins.

Bilens motor kan derved bremses, ved at sætte en I/O pin høj, og derved vende strømmen igennem motoren.

(Eventuelt beskrivelse af komponenterne vi har brugt + at nævne det kan skade gearingen.)

% !TEX root = ../rapport.tex
\newpage
\subsection{Boost converter}
En boost converter er et analogt DC-DC kredsløb som har en højere udgangsspænding end indgangsspænding.
Idéen med en boost converter er at øge effekten hen over motoren og derved få højere acceleration og tophastighed. Ved 15V trækker motoren kun 0,5A og derved bruger den kun 7.5 Watt. Da vi har en strømforsyning der kan levere 30 Watt kan vi udnytte det meget mere ved at øge spændingen. Dette vil dog gøre vi ikke helt kan trække 30 Watt da man aldrig kan lave en 100 procent effektiv boost converter.  Ideén er at have en udgangsspænding på 30V, og hvis boost converteren kun er 85 procent effektiv giver det en strøm, I på:
\begin{equation}
15V \cdot 2A = 30 W
\end{equation}
\begin{equation}
30 W \cdot 0,85 = 25,5 W
\end{equation}
\begin{equation}
I = \frac{25,5 W}{30 V} = 0,85A
\end{equation}
Hvilket burde være mere end rigeligt til at drive motoren langt hurtigere end med 15V og 0,5A

\begin{figure}[ht]
    \centering
    \includegraphics[width=0.8\linewidth]{kapitler/billeder/BoostConverter.png}
    \caption{Skematisk tegning af en boost-converter}
    \label{fig:boostconvert}
\end{figure}

På figur \ref{fig:boostconvert} ses et forsimplet kredsløb af en boost converter. Idéen er at når switchen S, sluttes bliver der en magnetisk kraft opladet i spolen L. Når switchen bliver brudt vil kraften fra spolen oplade kondensatoren, men da strømmen i en spole ikke kan ændres momentant vil der også løbe strøm fra spændingskilden igennem spolen og derved vil der opnås en højere spænding over kondensatoren C.

Da dette kredsløb gerne skulle være selvkørende bruges der ikke en switch, men en MOSFET som skal aktiveres af et højfrekvent signal. Det er vigtig at frekvensen er så høj at spolen ikke når at gå i mætning, som vil sige at det ikke når det punkt hvor virker som en kortslutning, da kredsløbet så ville være en direkte kortslutning fra vores spændingskilde til ground.

Det er vigtigt at udvælge de rigtige komponenter til dette kredsløb da det er et højfrekvent kredsløb hvor der løber store strømme. En schottky-diode benyttes da den kan switche ved langt højere frekvenser end en almindelig diode og har et lavere spændingstab henover den og dermed får vi større effektivitet. Kondensator skal have så stor kapitans som mulig(selvfølgelig inden for rimelighedens grænser) og kunne tåle høje spændinger hen over den. MOSFET'en som bliver brugt som switch skal kunne switche hurtigt og kunne klare at der løber stor strøm igennem den uden at blive alt for varm.

En af de problemer ved at bygge en boost converter som fast skal kunne levere 30V er an på motorens hastighed ændre motorens indre modstand sig, og derved ændrer spændingen sig også. Det er derfor nødvendigt at have en form for feedback der fortæller frekvensgeneratoren om den skal øge eller sinke frekvensen for at outputtet bliver som forventet. Der findes IC-kredse der gør dette automatisk, men da gruppen ikke kan stå indenfor hvordan de IC-kredse virker er der blevet valgt ikke at bruge dem i  bilen, og derved ville det blive for stor en udfordring at bygge boost converteren i forhold til den tid der var til projektet.  


% !TEX root = ../rapport.tex

\subsection{Sensorer}


\subsubsection{Stregsensor}
Formålet med stregsensoren er at detektere målstregen. Dette gøres optisk, da selve banen er sort, og målstregen er hvid. Da hvid reflektere lys meget bedre end sort får vi et signal der er let at komparere.

Stregsensoren er bygget op som vist på ref diagram. Den fungere ved at dioden lyser ned på banen, det lys der bliver reflekteret modtager fototransistoren. Ved målstregen har den et output på 4.8 V og på resten af banen 2.5 V. Til at komparere er der blevet brugt en schmitt-trigger, således at der kommer en hysterese. Dette er valgt for at undgå at registrere målstregen flere gange i det vi passere den. Reference spændingen ved det inverterende ben på schmitt-triggeren er sat til 3.3 V og hysteresen som vist på figur nedenunder.
\\
Indsæt hysterese kurv*
\\
Stregsensoren sender et signal, ved målstregen på 4.9 V som  mikrocontrolleren modtager som et logisk højt signal. Ved placering over den sorte del af banen sender stregsensoren et signal på 0.03V som mikrocontrolleren opfatter som et logisk lavt signal.

Ved placering over metalskinnerne på banen sender sensoren også et logisk højt signal, for at undgå at modtage den støj, er sensoren placeret bag venstre forhjul, det gør også at mikrocontrolleren modtager signalet om målstregen hurtigere end hvis den var placeret bagerst på bilen.

Da det er fordelagtig at havde så lidt vægt i bilen, og for at skabe så meget plads som muligt til blandt andet elektromagneten, er det blevet lavet med SMD-komponenter.


\subsubsection{Motor-Omdrejningstæller}
\label{motor-omdrej}

For at måle distancer, så bilen ved hvor langt der er imellem sving på banen, er der blevet lavet en omdrejningstæller. Til dette kan samme kredsløb som stregsensoren anvendes, idet der er blevet fremstillet en rotations enkoder, der er fastspændt på motorens akse. Dog er modstanden under transistoren ændret få at gøre sensoren mindre følsom.


\begin{figure}[ht]
    \centering
    \includegraphics[width=0.4\linewidth]{kapitler/billeder/dekoder.png}
    \caption{3D-tegning af optisk dekoder}
    \label{fig:dekoder}
\end{figure}


Det vil sige at omdrejningstælleren vil sende et logisk højt signal til mikrocontrolleren når den er ved et af hullerne i  rotationsenkoderen, og et logisk lavt signal resten af tiden. En puls fra Omdrejningstælleren vil svare til 1/6 af en motoromdrejning.
Da denne sensor kommer til at blive brugt i flere frekvensområder alt efter motorens hastighed, er det vigtigt at komparatoren ikke har en for langsom slewrate.

Den komperator der er blevet brugt i kredsløbet er en LM311 har ved normal opsætning en respons tid på 165 ns.

\begin{equation}
\frac{5 V}{168 ns} = 3 \cdot 10^7 \frac{v}{s}
\end{equation}

Signalet ved motor omdrejninger ved 2 kHz vil give 12 pulser, det giver et signal der skal går fra 5 til 0 volt med 12 kHz.
\begin{equation}
\frac{5 V}{(12 \cdot 10^-3)s} = 416.7 v/s
\end{equation}

Det kan derfor ses at selv hvis bilen akse drejer med 2 kHz kun sensoren følge med.


Kilde til datasheet: http://www.ti.com/lit/ds/symlink/lm311.pdf



\subsection{Delkonklusion}
De fysiske ændringer der er lavet på bilen optimere næsten alle krav der var stillet inden udviklingen af de enkelte ting blev påbegyndt.
Minimering af vægt blev gjort ved at fjerne alt unødvendigt plastik fra bilen og printe et nyt lettere chassis.
Justerbar downforce blev indført vha. en elektromagnet. Tyngdepunktet blev sænket med det nye chassis der samtid afstivet bilen.
Bremselængde blev kraftig minimeret vha. h-broen dog skal man være opmærksom på at tandhjulene ikke kan holde til at man give motoren for meget fart i modsat retning af køreretningen.
2Accelerationen og tophastigheden blev ikke optimeret da en boost converter blev vurderet til at være for tidskrævende i forhold til projektet.
