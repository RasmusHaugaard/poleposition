% !TEX root = ../rapport.tex

\section{Fysiske modifikationer og hardware}
Som nævnt tidligere skal bilen både optimeres fysisk og ved hjælp af software. I dette kapitel bliver der sat fokus på hvordan bilen er blevet optimeret fysisk, og hvilket hardware der er blevet udviklet til bilen. 
For at optimere bilens omgangstider mest muligt er der også fortaget fysiske modifikationer på bilen. Som beskrevet i opgaveformulering var der også krav til at der skulle bruges enten en elektromagnet som aktuator eller en hall-sensor. Der er blevet foretrukket en elektromagnet til at skabe yderligere downforce. De ting der blandt andet er med til at give en optimeret omgangstid er:
\begin{itemize}
\item Minimering af vægt.
\item Justerbar downforce.
\item Lavere tyngdepunkt.
\item Afstivning af chassis.
\item Minimering af bremselængde.
\item Maksimering af acceleration og tophastighed.
\end{itemize}


% !TEX root = ../rapport.tex
\subsection{Fysiske modifikationer}

For at minimere vægten af bilen er alt unødvendigt plastik blevet fjernet fra bilen. Grunden til dette kan beskrives vha. Newtons 2. lov. Bilen bevæger sig fremad med en kraft F, og bilen har en masse m. 
\begin{equation}
F=m \cdot a
\end{equation}
\begin{equation}
a = \frac{F}{m}
\end{equation}
Accelerationen a, stiger derved jo mindre massen af bilen er. 

Downforce er ekstremt brugbart i sving da det hjælper med at holde bilen på banen ved højere fart. Downforce er mindre brugbart på lige strækninger da bilen let bliver på banen og downforce faktisk vil sænke bilens acceleration og topfart. Derfor indføres der justerbar downforce med elektromagneten. I det 3D-printet chassis er der en indbygget holder til elektromagneten. Elektromagnet er placeret bagerst for at give kontra til den permamagnet der i forvejen sidder i bilen. Permamagneten er placeret foran og tæt på splitten for at trække splitten ned i banen. For uddybning af elektromagneten se sektion \ref{Elektromagnet}.

\subsubsection{3D-printet chassis}

For at kunne hurtigt gennem sving kræver det et lavt tyngdepunkt i bilen, jo lavere tyngdepunktet er jo hurtigere kan bilen klare at køre gennem et sving uden at kæntre. Dette fænomen kan ses på figur \ref{fig:bilsving} nedenfor.
\begin{figure}[ht]
    \centering
    \includegraphics[width=0.8\linewidth]{kapitler/billeder/bilsving.png}
    \caption{Skitse af bilen i et venstresving med to forskellige tyngdepunkter}
    \label{fig:bilsving}
\end{figure}
På bil 1 er tyngdepunktet (den blå prik) lavt og derved har den en kortere arm (den grønne streg) fra tyngdepunktet til omdrejningspunktet (den røde prik). Omdrejningsmomentet(den lilla pil) kan beskrives som:
\begin{equation}
\tau = F \cdot L
\end{equation}
Hvor $\tau$ er omdrejningsmomentet, L er armen (den grønne streg) og F er kraften der påvirker den (den sorte pil). Der antages at der skal det samme omdrejningsmoment ($\tau$) til at bilen kæntre.
\begin{equation}
F = \frac{\tau}{L}
\end{equation}
Derved kan det ses at når armen, L, bliver større skal der mindre kraft F til bilen kæntre. 
Ved at fjerne affjedringen optimeres tyngdepunktet også da karrosseriet er i sin laveste tilstand hele tiden, og ikke kun når fjedrene er trykket sammen.

Det udleveret chassis på racerbilen er meget vakkelvornt, og kan vrides let. For at optimere dette er der blevet 3D-printet et nyt forstærket chassis. Det chassis er stærkere end det gamle og giver lettere plads til at montere sensorer og aktuatore i bilen. Det nye chassis er tegnet i AutoDesk Inventor og derefter 3D-printet med Makerbot. 3D-printet er printet i flere dele og derefter samlet med skruer. En fordel ved at printet er i flere dele er hvis én del går i stykker skal man ikke printe det hele igen. 3D-tegningen ses på figur \ref{fig:3Dtegning} nedenfor.

\begin{figure}[ht]
    \centering
    \includegraphics[width=0.7\textwidth]{kapitler/billeder/3Dtegning.jpg}
    \caption{3D-tegning af bilens chassis}
    \label{fig:3Dtegning}
\end{figure}


Det vil sige at den affjedring der følger med bilen er blevet fjernet og erstattet med det afstivet chassis. Grunden til man har affjedring i virkelige biler er for at beskytte både kører og motor, da man aldrig er sikker på asfaltens tilstand. Da banen racerbilen skal køre på er flad og der er ingen kører der skal beskyttes, vil det optimere bilens hastighed igennem sving uden at den kæntre, at fjerne affjedringen. På figur \ref{fig:racerchassis} nedenfor ses det 3D-printet chassis med motor og hjul på.

\begin{figure}[ht]
    \centering
    \includegraphics[width=0.7\textwidth]{kapitler/billeder/racerchassis.jpg}
    \caption{3D-printet chassis med motor og hjulakser}
    \label{fig:racerchassis}
\end{figure}

Det nye 3D-printet chassis er også udviklet til at ATmega-boardet kan skrues direkte på. Det nye chassis er blevet testet mod det gamle chassis i et 180 graders sving. Med det gamle chassis var den maksimale indgangshastighed bilen kunne køre ind i et sving uden at ryge af 1.9 m/s. Med det nye chassis kan bilen kører samme sving med en indgangshastighed på 2.6 m/s uden at ryge af.

% !TEX root = ../rapport.tex

\subsection{Elektromagnet}
\label{Elektromagnet}

%"Har lavet flere kommentare i  sån så alle i gruppen lige er opmæksomme på det, har gemt et indivuduelt tekst dokument så det kan genlæses, ellers slettes disse bare når i har læst dem "

Elektromagnetens formål er at give bilen ekstra nedadrettet kraft i svingende, således at bilen kan køre hurtigere rundt i svinget, og i sidste ende køre en hurtigere omgangstid. Se afsnit \ref{fysiske-modifik}. Dette kan gøres da racerbanen der køres på har 2 stål skinner i midten af banen.

Da det ikke er fordelagtig at havde ekstra downforce i løbet af længere lige strækninger, grundet den unødig ekstra friktion det vil skabe, skal magneten også kunne slås til og fra. Dette er blevet udført ved at styre elektromagneten med en mosfet der er koblet til mikrocontrolleren. Det bliver ud fra mikrocontrolleren sendt en puls med ca. 30 kHz til mosfett’en og da spolen vil modvirke momentane ændringer, vil den opfatte det som en DC strøm, der varigere efter ønsket duty cykle. Dette gør at man kan undgå at spolen bliver for varm, men også at intensiteten af elektromagneten kan kontrolleres.
%"Har snakket med Jan og der kommer kun kompleksemodstande ved en sinus signal, men vi sender det der hedder transient signal / firkantet signal, og spolen modvirker jo de her momentane ændringer. Slet når læst "

Berigninger på at mosfetten ikke bliver for varm kan ses på bilag \ref{subsec:mosfet}.

\subsubsection{Størrelse}

Dette medføre at elektromagneten først og fremmest er begrænset af bilens dimensioner, da der ønskes at lave den så kraftig som mulig. I det special lavet chasiss er der et frirum på bredden 1.5cm, højden 1.5 cm og længden 1.5 cm. Altså et total volume for hele elektromagneten på 3.375 $cm^3$
%"Ved godt det ikke helt er det, men i stedet for at fylde rapporten ud med meget tid og plads fyldende beregninger på noget så kedligt som volume og for at holde fokus på det vigtige om elektromagneten er dette valgt. slettes når læst"

Elektromagnetens kerne er af stål, og har en total volumen på 0.645 $cm^3$ giver det os resterne 2.73 $cm^3$ til spolen.
%"kernens 2 ben=2*1.5*1.5*0.1 cm + kernen del hvorpå spolen sidder 0.1*1.3*1.5.  slettes når læst "

Dette kan bruges til at estimere hvor mange vindinger N, vi kan havde med en ledning med tværsnitsareal $A_w$, med en spole med gennemsnit radius på r som er 0.45 cm.  $l_{spole}$ er længden af ledningen viklet om spolen.
%"Gennemsnit omkreds da omkredsen stiger linært i takt med der kommer viklinger. 0.3cm for kernen, spole slutter ved 0.6cm det giver gns på 0.45 cm….. *Givet at viklingerne ligger helt tæt.. slettes når læst "

\begin{equation}
Vol_{max} =l_{spole} \cdot A_w = 2\pi \cdot r \cdot N \cdot A_w
\end{equation}
\begin{equation}
N= \frac{Vol_{max}}{2\pi \cdot r \cdot A_w}
\end{equation}

Til elektromagneten i dette projekt blev der brugt en 0.2 mm kobberledning, da spolen let spindes på maskine uden at lakeringen blev skrabet af og kortslutninger fremkom. Dog et skridt der let kunne optimere elektromagnetens ydeevne da en tyndere ledning ville resultere i flere viklinger. Dog skal man være opmærksom på man ikke sender for meget strøm igennem ledningen end den kan holde til.
%"Slettes evt. hvis vi ikke ønsker at skulle snakke om resitivitet til eksamen mm. Har dog ikke nogle gode nok beregninger til at kunne tages med i rapporten."

Teoretisk vil dette medføre en spole om kernen med følgende antal vindinger:

\begin{equation}
N=\frac{2.73*10^{-6} m^3}{0.0045m*\pi*2*((0.0002m)^{2}*\pi)} = 769 vindinger
\end{equation}
I praksis blev der dog kun plads til 600 vindinger.

\subsubsection{Kraft}

Den kraft denne elektromagnet vil kunne trække bilen ned med kan estimeres ved først at kigge på den magnetiske energi $W_m$, der vil blive opladt i en mængde materiale $ \Delta V$ med en konstant permeabilitet $ \mu $. \cite{FysBog}

\begin{equation}
W_m =\frac{B^2}{2 \mu} \Delta V
\end{equation}

Hvis man integrere energi densiteten over volumen af hele det magnetiske kredsløb og antager at B-feltet bliver holdt konstant igennem tværsnits arealet A på elektromagneten, medføre det at:

\begin{equation}
\Phi=B*A
\end{equation}
\begin{equation}
B=\frac{\Phi}{A}
\end{equation}

\begin{equation}
W_m = \frac{1}{2} \int B^2 \frac{dv}{\mu}  = \frac{1}{2} \Phi^2 \int \frac{1}{A^2 \cdot \mu} dV
\end{equation}

Det gælder for et magnetisk kredsløb at den resulterende flux $\Phi$ kan regnes som:

\begin{equation}
\Phi \simeq \frac{NI}{R_m} \simeq  \frac{NI}{\frac{1}{A\mu_0}}
\end{equation}

$N \cdot I$ er ampere vindinger, $R_m$ er kredsløbets totale reluktans, $ \mu $ er kredsløbets permeabilit og l er længden af kredsløbet. Det gælder for kredsløbet at:

\begin{equation}
\Delta V =A \cdot dl
\end{equation}

Det medføre at kredsløbets energi kan skrives som:

\begin{equation}
W_m = (NI)^2/(2(R_m)^2) \int \frac{dl}{A \cdot \mu}
\label{fig:wm}
\end{equation}

Det gælder at den samlede reluktans er summen af alle reluktanserne, i kredsløbet.

\begin{equation}
R_m = \int \frac{dl}{A \cdot \mu}
\label{fig:rm}
\end{equation}

Sammensættes lignin \ref{fig:rm} og \ref{fig:wm} fås:

\begin{equation}
W_m = \frac{(NI)^2}{2R_m}
\label{wmdone}
\end{equation}

Energi balancen i kredsløbet er.
$mekanisk arbejde + magnetisk energi =Elektrisk energi \cdot tid$

\begin{equation}
F dx + dW_m = iv dt
\label{Fdx}
\end{equation}

Givet at strømmen i spolen kan antages som konstant vil den elektrisk energi skrives som.

\begin{equation}
iv=IN \frac{d\Phi}{dt} =(NI)^2  \frac{d\frac{1}{R_m}}{dt}
\label{iv}
\end{equation}


Sammensættes ligning \ref{wmdone}, \ref{iv} og \ref{Fdx} fås det:

\begin{equation}
F dx+\frac{(NI)^2}{2d} \frac{1}{R_m} =(NI)^2  d \frac{1}{R_m}
\end{equation}

Ud fra dette kan det konstateres at halvdelen af den totale elektriske energi bliver omdannet til mekanisk arbejde.

\begin{equation}
F = \frac{dR_m}{dx} (-\frac{1}{2} (\frac{(NI)}{Rm} )^2 )
\end{equation}

Hvor x er elektromagnetens længde fra skinnen ganget med 2, da vi har en hesteformet magnet.

Den samlet reluktans er summen af alle kredsløbets reluktanser, I dette tilfælde vil der være reluktans fra selve elektromagneten $R_k$, fra luftgabet $R_l$ og fra skinnerne i racerbanen $R_s$.

Som det kan ses på figuren fornede.

\begin{figure}[ht]
	\label{fig:Mkreds}
	\centering
	\includegraphics[width=1\textwidth]{kapitler/billeder/EMagnet.jpg}
	\caption{Elektromagnetisk kredsløbs analogi}
	\end{figure}

%Tegnes Lørdag eller søndag beklager ventetiden.

\begin{equation}
\label{rm1}
R_m =R_k + R_s + R_l = \frac{l_{kerne}}{\mu_{r.staal} \cdot \mu_0 \cdot A_{kerne} } + \frac{l_{skinne}}{\mu_{r.staal} \cdot \mu_0 \cdot A_{skinne} }  + \frac{l_{kerne}}{\mu_0 \cdot A_{luftgab} }
\end{equation}

Hvor l står for henholdsvis længden af kernen, skinnen og luftgabet. A for tværsnitsarealet,og $ \mu_{r.staal} $ for den relative permeabilitet til forhold luftens permeabilitet $\mu_0$.  For at forsimple udtrykket kan den totale reluktans sammenlignes med luftgabets reluktans en værdi  der fremover bliver kaldt z.

\begin{equation}
z = \frac{R_m}{R_l} = \frac{R_k}{R_l} +\frac{R_s}{R_l} + 1 = \frac{
\frac{l_{kerne}}{l_{luftgab}} }
	{\mu_{r.staal} \cdot \frac{A_{kerne}}{A_{luftgab}} }
+
\frac{
	\frac{l_{skinne}}{l_{luftgab}} }
{\mu_{r.staal} \cdot \frac{A_{skinne}}{A_{luftgab}} }
+ 1
\end{equation}
%"Motherfucker equation... i know, men den er meget væsentlig og burde ikke skrives kortere"

Ved en lille fluxspredning vil det kunne antages at.

\begin{equation}
A_{kerne} \simeq A_{luftgab}
\end{equation}
\begin{equation}
A_{skinne} \ll A_{luftgab}
\end{equation}

Og idet at der tilføjes elektrisk stål til elektromagnetens kerne og det gælder at:
%"leder stadig efter en god kilde på permabiliteter... en der er bedre end wiki..."

\begin{equation}
\mu_{staal} \ll \mu_{Estaal}
\end{equation}
\begin{equation}
  \frac{\frac{l_{kerne}}{l_{luftgab}}}{
  	{\mu_{r.Estaal} \cdot \frac{A_{kerne}}{A_{luftgab}} }} \ll \frac{
	\frac{l_{skinne}}{l_{luftgab}} }
{\mu_{r.staal} \cdot \frac{A_{skinne}}{A_{luftgab}} }
\end{equation}

Medføre dette at vi vælger kernen reluktans kan negligere, således kan det redfærdigøres at omskrive ligningen til følgende.

\begin{equation}
z =
\frac{
	\frac{l_{skinne}}{l_{luftgab}} }
{\mu_{r.staal} \cdot \frac{A_{skinne}}{A_{luftgab}} }
+ 1
\end{equation}

Hvis skinnens reluktans undersøges nærmere ses følgende.

%" Rs Ganges og divideres med Lluftagab/Aluftab, da det en korrekt matematisk indgreb der ikke ændre værdien af Rs men gør vi kan sammenligne Rs og Rl… slettes når læst. "

\begin{equation}
R_s =  \frac{l_{skinne}}{\mu_{r.staal} \cdot \mu_0 \cdot A_{skinne} }
=
\frac{
	\frac{l_{skinne}}{l_{luftgab}} }{\mu_{r.staal} \cdot \frac{A_{skinne}}{A_{luftgab}} }\cdot \frac{l_{luftgab}}{\mu_0 \cdot A_{luftgab}}=\frac{
	\frac{l_{skinne}}{l_{luftgab}} }
	{\mu_{r.staal}\cdot\frac{A_{skinne}}{A_{luftgab}}}\cdot R_{luftgab}
\end{equation}


Da elektromagneten er sænket så meget som muligt holdes den ca. 0.5 mm over skinnen, og skinnens tværsnit fås til $2mm \cdot 4mm$ kan følgende observeres.

\begin{equation}
R_s =
\frac{\frac{15 mm}{1 mm} }
{ 500 \cdot R_{luftgab} }= 0.06\cdot R_{Luftgab}
	\end{equation}

Dette betyder at for dette kredsløbs funktion kan det med rimelighed antages at:

\begin{equation}
R_m \simeq R_l
	\end{equation}

Dette medføre at ligning \ref{rm1} kan skrives som:

\begin{equation}
\mid F \mid = \frac{(NI)^(2) \cdot \mu_0 \cdot A_luftgab } {2 \dot x^2}
\end{equation}

Det kan derfor ses at for at optimere magneten vil de mest betydende led være amperevindinger om magneten og længden af luftgabet.

For magneten der er anvendt til projektet vil den teoretiske kraft, magneten ville genere ved 1 A, placeret 0.5 mm over banen være: %(OBS x=2*placering over banen… slet når læst)

\begin{equation}
\mid F \mid = \frac{(600 amperevindinger)^2 \cdot \mu_0 \cdot 1.5 \cdot 10^{-5} m^2 } {2 \dot (0.001m)^2}
\end{equation}

I praksis kunne magneten ved 0.5 mm placering over skinnen dog kun trække  1.69 N. Det tyder at alle antagelserne gjort har haft større indflydelse end forventet.

\subsubsection{Forsøg}
Da permabiliteten for stål er relativ lav tilforhold fks. Elektrisk stål, er der blevet forsøg at tilsætte elektrisk stål til elektromagnetens kerne. Efter forsøg hvor magneten blev placeret på skinnen, kunne det konkluderes at tilføje Elektriskstål til magnetens kerne kunne forøge den kraft elektromagneten kunne trække sig mod skinnen med.
Ved forsøget var magneterne placeret på skinnen. Se figur \ref{fig:Magnet}

\begin{figure}[ht]
	\centering
	\includegraphics[width=1\textwidth]{kapitler/billeder/Magnet.png}
	\caption{Gennemsnitsresultat fra forsøg}
	\label{fig:Magnet}
\end{figure}


\subsubsection{Delkonklusion}
Da amperevindinger efter vores formel skulle havde stor betydning for elektromagnetens kraft er kernen på elektromagneten lavet af et tyndt stykke stål der kun er 1 mm tykt for at få plads til så mange vindinger som muligt. Da ståls permeabilitet ikke er lige så god som for eksempel jern eller elektriskståls, blev der ekspermiteret med at tilføje elektrisk stål til kernen, hvilket gav gode resultater. Den endelige magnet placeret i bilen endte med at trække 1.69 N.

Udover amperevindinger havde magnetens placering fra skinnerne på racerbanen også stor betydning. Dette blev der taget højde for ved at placere elektromagneten så tæt på skinnen som muligt. Ud fra tests, blev en placering på 0.5 mm over banen den bedste. Her blev der ikke skabt for meget bremsende friktion og med lidt tape på undersiden af magneten kortsluttede banen ikke ved ujævnheder i banen.

% !TEX root = ../rapport.tex

\subsection{Bremsefunktion (H-bro)}

Ligesom acceleration, er bremselængden en vigtig faktor for at opnå en god omgangstid.
Når bilen nærmer sig et sving, handler det om at komme så tæt på svinget som muligt, før at bilen
skal begynde at bremse. Dette kræver en effektiv bremse, så bremselængden minimeres og omgangstiden effektiviseres.
For at bremse bilen så effektivt som muligt, benyttes princippet bag en H-bro. Dette princip er illusteret
på figur \ref{fig:hbro}, nedenfor.

\begin{figure}[ht]
    \centering
    \includegraphics[width=1\textwidth]{kapitler/billeder/hbro.png}
    \caption{Viser princippet bag en H-bro, og hvordan den benyttes i dette projekt.}
    \label{fig:hbro}
\end{figure}

Som illusteret på figur \ref{fig:hbro}, kan man bruge en H-bro til at vende strømmen igennem motoren.
Ved at vende strømmen igennem motoren i få millisekunder, vil motoren prøve at bakke,
og bilen vil bremse meget effektivt. Figuren viser det to scenarier, hvor strømmen til højre er vendt
og strømmen til venstre ikke er vendt.

H-broen konstrueres med et dobbelt relæ, som vender de to kontakter når spolen inducere en strøm.
Spolen kræver en høj aktiveres spændning for at trække den nødvendige strøm
og kredsløbet bygges derfor op med en transistor.
Transistoren kan aktiveres med en meget lille strøm og spændning, og kan derfor aktiveres
af ATmega'ens I/O pins.

Bilens motor kan derved bremses, ved at sætte en I/O pin høj, og derved vende strømmen igennem motoren.

(Eventuelt beskrivelse af komponenterne vi har brugt + at nævne det kan skade gearingen.)

% !TEX root = ../rapport.tex
\newpage
\subsection{Boost converter}
En boost converter er et analogt DC-DC kredsløb som har en højere udgangsspænding end indgangsspænding.
Idéen med en boost converter er at øge effekten hen over motoren og derved få højere acceleration og tophastighed. Ved 15V trækker motoren kun 0,5A og derved bruger den kun 7.5 Watt. Da vi har en strømforsyning der kan levere 30 Watt kan vi udnytte det meget mere ved at øge spændingen. Dette vil dog gøre vi ikke helt kan trække 30 Watt da man aldrig kan lave en 100 procent effektiv boost converter.  Ideén er at have en udgangsspænding på 30V, og hvis boost converteren kun er 85 procent effektiv giver det en strøm, I på:
\begin{equation}
15V \cdot 2A = 30 W
\end{equation}
\begin{equation}
30 W \cdot 0,85 = 25,5 W
\end{equation}
\begin{equation}
I = \frac{25,5 W}{30 V} = 0,85A
\end{equation}
Hvilket burde være mere end rigeligt til at drive motoren langt hurtigere end med 15V og 0,5A

\begin{figure}[ht]
    \centering
    \includegraphics[width=0.8\linewidth]{kapitler/billeder/BoostConverter.png}
    \caption{Skematisk tegning af en boost-converter}
    \label{fig:boostconvert}
\end{figure}

På figur \ref{fig:boostconvert} ses et forsimplet kredsløb af en boost converter. Idéen er at når switchen S, sluttes bliver der en magnetisk kraft opladet i spolen L. Når switchen bliver brudt vil kraften fra spolen oplade kondensatoren, men da strømmen i en spole ikke kan ændres momentant vil der også løbe strøm fra spændingskilden igennem spolen og derved vil der opnås en højere spænding over kondensatoren C.

Da dette kredsløb gerne skulle være selvkørende bruges der ikke en switch, men en MOSFET som skal aktiveres af et højfrekvent signal. Det er vigtig at frekvensen er så høj at spolen ikke når at gå i mætning, som vil sige at det ikke når det punkt hvor virker som en kortslutning, da kredsløbet så ville være en direkte kortslutning fra vores spændingskilde til ground.

Det er vigtigt at udvælge de rigtige komponenter til dette kredsløb da det er et højfrekvent kredsløb hvor der løber store strømme. En schottky-diode benyttes da den kan switche ved langt højere frekvenser end en almindelig diode og har et lavere spændingstab henover den og dermed får vi større effektivitet. Kondensator skal have så stor kapitans som mulig(selvfølgelig inden for rimelighedens grænser) og kunne tåle høje spændinger hen over den. MOSFET'en som bliver brugt som switch skal kunne switche hurtigt og kunne klare at der løber stor strøm igennem den uden at blive alt for varm.

En af de problemer ved at bygge en boost converter som fast skal kunne levere 30V er an på motorens hastighed ændre motorens indre modstand sig, og derved ændrer spændingen sig også. Det er derfor nødvendigt at have en form for feedback der fortæller frekvensgeneratoren om den skal øge eller sinke frekvensen for at outputtet bliver som forventet. Der findes IC-kredse der gør dette automatisk, men da gruppen ikke kan stå indenfor hvordan de IC-kredse virker er der blevet valgt ikke at bruge dem i  bilen, og derved ville det blive for stor en udfordring at bygge boost converteren i forhold til den tid der var til projektet.  


% !TEX root = ../rapport.tex

\subsection{Sensorer}


\subsubsection{Stregsensor}
Formålet med stregsensoren er at detektere målstregen. Dette gøres optisk, da selve banen er sort, og målstregen er hvid. Da hvid reflektere lys meget bedre end sort får vi et signal der er let at komparere.

Stregsensoren er bygget op som vist på ref diagram. Den fungere ved at dioden lyser ned på banen, det lys der bliver reflekteret modtager phototransistoren. Ved målstregen har den et output på 4.8 V og på resten af banen 2.5 V. Til at komparere er der blevet brugt en schmitt trigger, således at der kommer en hysterese. Dette er valgt for at undgå at registrere målstregen flere gange i det vi passere den. Reference spændingen ved det inverterende ben på schmitt-triggeren er sat til 3,3 V og hysteresen som vist på figur nedenunder.
\\
Indsæt hysterese kurv*
\\
Stregsensoren sender et signal, ved målstregen på 4.9 V som  mikrocontrolleren modtager som et logisk højt signal. Ved placering over den sorte del af banen sender stregsensoren et signal på 0.03V som mikrocontrolleren opfatter som et logisk lavt signal. 

Ved placering over metalskinnerne på banen sender sensoren også et logisk højt signal, for at undgå at modtage den støj, er sensoren placeret bag venstre forhjul, det gør også at mikrocontrolleren modtager signalet om målstregen hurtigere end hvis den var placeret bagerst på bilen. 

Da det er fordelagtig at havde så lidt vægt i bilen, og for at skabe så meget plads som muligt til blandt andet elektromagneten, er det blevet lavet med SMD-komponenter. 


\subsubsection{Motor-Omdrejningstæller}

For at måle distancer, så bilen ved hvor langt der er imellem sving på banen, er der blevet lavet en omdrejningstæller. Til dette kan samme kredsløb som stregsensoren anvendes, idet der er blevet fremstillet en rotations enkoder, der er fastspændt på motorens akse. Dog er modstanden under transistoren ændret få at gøre sensoren mindre følsom.   


\begin{figure}[ht]
    \centering
    \includegraphics[width=0.4\linewidth]{kapitler/billeder/dekoder.png}
    \caption{3D-tegning af optisk dekoder}
    \label{fig:dekoder}
\end{figure}


Det vil sige at omdrejningstælleren vil sende et logisk højt signal til mikrocontrolleren når den er ved et af hullerne i  rotationsenkoderen, og et logisk lavt signal resten af tiden. En puls fra Omdrejningstælleren vil svare til 1/6 af en motoromdrejning.
Da denne sensor kommer til at blive brugt i flere frekvensområder alt efter motorens hastighed, er det vigtigt at komparatoren ikke er ?begrænset af dens slewrate eller bandwith. ? 


??
Den komperator der er blevet brugt i kredsløbet er en LM311 har ved normal opsætning en respons tid på 165 ns fra høj til lav og 115 ns fra lav til høj.

\begin{equation}
\frac{1 V}{280 ns} = 3.571.429 \frac{v}{s}
\end{equation}

Hvor signalet ved motor omdrejninger ved 2 kHz vil give 12 pulser, det giver et signal der går fra 5 til 0 og 0 til 5 volt med 12 kHz.
\begin{equation}
\frac{10 V}{(12 \cdot 10^-3)s} = 833 v/s
\end{equation}


?? for vagt?? 



Kilde til datasheet: http://www.ti.com/lit/ds/symlink/lm311.pdf


?Hastighed?

inkoder som hver puls fra inkoderen svare til en 1/6 af en motor omdrejning. 


\subsection{Delkonklusion}
De fysiske ændringer der er lavet på bilen optimere næsten alle krav der var stillet inden udviklingen af de enkelte ting blev påbegyndt. Minimering af vægt blev gjort ved at fjerne alt unødvendigt plastik fra bilen og printe et nyt lettere chassis. Justerbar downforce blev indført vha. en elektromagnet. Tyngdepunktet blev sænket med det nye chassis der samtid afstivet bilen. Bremselængde blev kraftig minimeret vha. h-broen dog skal man være opmærksom på at tandhjulene ikke kan holde til at man give motoren for meget fart i modsat retning af køreretningen. Accelerationen og tophastigheden blev ikke optimeret da en boost converter blev vurderet til at være for tidskrævende i forhold til projektet.