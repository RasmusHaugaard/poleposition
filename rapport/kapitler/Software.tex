% !TEX root = ../rapport.tex

\section{Software}
Nu er bilen optimeret mest muligt rent hardwaremæssigt. For at kunne kører den hurtigst muligt omgangstid skal der også udvikles noget sofware. Softwaren skal henholdsvis bruges til at:
\begin{itemize}
\item Kommunikere med bluetooth.
\item Kommunikere med sensorer.
\item Aflæsning af optisk dekoder og stregdetektor.
\item Lave et map af den ukendte bane.
\item Styre motoren i forhold til det gemte map.
\end{itemize}
Alt software til bilen er udviklet i assembly.

% !TEX root = ../rapport.tex
\subsection{Udviklingsmiljø}

Softwaredelen er en stor del af opgaven. Der er derfor lagt en del energi i at optimere gruppens udviklingsmiljø.

Fem personer har skulle skrive moduler i assembly-kode, der skulle virke med andre moduler, samt testprogrammer, der inkluderede disse moduler. Disse testprogrammer skulle flashes ned på mikrocontrolleren.

\subsubsection{Fildeling}
Det var nødvendigt, gruppens medlemmer kunne dele kode med hinanden. Dropbox og Github blev overvejet.
Dropbox forsøger at holde gruppens medlemmers filer synkroniseret.
Det gør det nemt og hurtigt at arbejde med.
Github tilbyder et online arkiv med versionsstyring, man aktivt skal hente fra og uploade til.

Grundet et ønske om, man kunne isolere sig fra ændringer i perioder og teste egen kode uden risiko for, ændringer fra andre medlemmer medfulgte fejl, samt sikkerheden i versionsstyringen, blev Github valgt til at hoste al kode samt rapporten.
Dropbox blev valgt til at hoste diverse filer som datablade, test videoer, billeder, 3D filer med mere.

\subsubsection{.filedef}
Assembleren avra har direktivet \mbox{``.def''} til at tildele registre navne. For eksempel, \mbox{``.def temp = R16''}.
For at kunne skrive moduler, der var så uafhængige af hinanden som muligt og kunne tildele registre navne begrænset af konteksten af modulet, blev der skrevet en preprocessor med et nyt direktiv, \mbox{``.filedef''}. For eksempel, \mbox{``.filedef temp = R16''}. Til forskel for \mbox{``.def''} vil \mbox{``.filedef''} som navnet antyder kun gælde i den fil, definitionen eksisterer. Preprocessoren er skrevet i Node.js og køres før assembleren i build setuppet.

\begin{mdquote}
	Note: Node.js er en cross-platform server-side JavaScript runtime. Node.js' ``package ecosystem, npm, is the largest ecosystem of open source libraries in the world'' TODO: kilde. (Meget forvirrende skrevet)
\end{mdquote}

\subsubsection{Build setup}
For nemt at kunne builde på både osx, linux og windows er build setuppet skrevet i Node.js.


% !TEX root = ../rapport.tex
\subsection{Bootloader}
Programmerstikket til mikroprocessoren er ikke nem at tilgå, når bilen er fuldt monteret. For at kunne iterere hurtigt og undgå at skulle skille og samle bilen ad hele tiden, blev der udviklet en bootloader.

\subsubsection{Protokol}
Bootloaderen styrer bluetoothkommunikationen og indeholder en overordnet kommunikationsprotokol.
Den første byte, bootloaderen modtager, afgør, hvordan bootloaderen skal håndtere de næstkommende bytes.
85: Set. 85 Gemmes som den første byte i inputbufferen. De to næstkommende gemmes også, før der hoppes til kommando interruptet.
87: Ping. Bootloaderen sender 87 tilbage med det samme.

For at undgå at skulle opdatere bootloaderen, hvis der blev implementeret andre kommandoer i applikationsprotokollen, blev der implementeret en variabel kommando, 86.
Den næstkommende byte beskriver længden, n, af kommandoen. De derefter næstkommende, n, bytes gemmes i inputbufferen, før der hoppes til kommando interruptet.
Bemærk:
[85, X, X] og [86, 3, 85, X, X] er ekvivalænte.
[87] er ping kommandoen til bootloaderen. [86, 1, 87] er en kommando [87] til applikationen.

Når bootloaderen mener, der er en hel kommando (liste af bytes) klar til applikationen, hoppes til 0x2A, en udvidelse af interrupt vektor tabellen.

% !TEX root = ../rapport.tex
\newpage

\subsection{I2C kommunikation}
Til at kommunikere med de digitale komponenter, herunder vores udleveret accelerometer og gyroskop, anvendes der en kommunikationsprotokol kaldet I2C eller TWI (”two-wire interface”). Navnet ”two-wire interface” kommer af, at hele kommunikationen foregår over kun 2 ledninger, men hvor der samtidig er mulighed for at kommunikere med mange komponenter.

\begin{figure}[ht]
    \centering
    \includegraphics[width=0.8\textwidth]{kapitler/billeder/i2c-princip.png}
    \caption{Viser en skitse af hvordan opsætningen kan laves mellem
én ”master” og flere ”slaver”, kun ved brug af to forbindelser.}
    \label{fig:i2cprincip}
\end{figure}


De to forbindelser der bruges til vores TWI kommunikation hedder SCL (”Serial Clock Line”) og SDA (”Serial Data Line”). SCL er en clock der sættes af master, for at både master og slaven snakker og lytter ved en fælles frekvens. SDA er den forbindelse, hvor alt data bliver overført både fra slaven til master, men også fra master til slaven. Denne form for kommunikation kaldes for halv-duplex, og fungere ligesom en walkie talkie, hvor der kun er én der snakker af gangen, imens den anden lytter.

\subsubsection{Kommunikationsprotokol}

Kommunikationen mellem master og slave, foregår ud fra en bestemt protokol, som er opgivet ved den enkelte slaves datablad. Denne protokol er den samme for accelerometeret og gyroskopet, og er illustreret på figur \ref{fig:i2conebyte}.

\begin{figure}[ht]
    \centering
    \includegraphics[width=1\textwidth]{kapitler/billeder/i2c-onebyte.png}
    \caption{Kommunikationsprotokol for gyroskopet via I2C}
    \label{fig:i2conebyte}
\end{figure}

Figur \ref{fig:i2conebyte} viser skridt for skridt, hvordan kommunikationsprotokollen er opbygget, hvis man vil læse data én gang fra én af vores to digital komponenter.

Protokollen indeholder følgende instruktioner for at modtage én data byte:

\begin{enumerate}
\item ST (”Start Bit”): Master sender et start bit til alle slaverne, for at starte kommunikationen.
\item SAD+W (”Slave Adress + Write”): Master sender en slave adresse på 7 bits, og det sidste bit beskriver om der skal læses eller skrives, i dette tilfælde skal der skrives. Adressen er specifik for hver komponent og fortæller hvilken slave, som resten af kommunikationen kommer til at foregå med. Det sidste bit indikere, at masteren vil skrive noget til slaven.
\item SAK (”Slave Acknowledge”): Slaven med den valgte adresse sender er ACK bit tilbage, som fortæller at den har hørt hvad masteren sagde, og gør klar til at læse næste instruktions.
\item SUB (”Sub adress”): Master sender herefter en underadresse. Denne underadresse er en 7 bit adresse, som fortæller hvilken register inde i den valgte komponent, der gerne vil adresseres.
\item SAK (”Slave Acknowledge”): Slaven sender endnu et SAK bit, og gør klar til næste instruktion.
\item SR (”Repeated Start”): Master sender herefter et gentagende start bit. Der skal altid sendes en start/repeated start instruktion, hver gang der skiftes mellem at læse eller skrive. Ved repeated start vedholdes forbindelsen til slaven, og slaven ved derfor allerede hvilken underadresse, der herefter skal læses fra.
\item SAD+R (”Slave Adresse + Read”): Master sender herefter igen slave adressen, samt én read bit, for at ændre at vi nu gerne vil læse på slaven, modsat at vores tidligere write, som skrev til slaven.
\item SAK (”Slave Acknowledge”):  Slaven sender en ACK bit.
\item DATA (”Data Byte”): Slaven sender efterfølgende en data byte til master.
\item NMAK (”Master Not Acknowledge”): Herefter sendes én NMAK fra master til slaven, for at indikere at der ikke skal læses mere data.
\item SP (”Stop Bit”): Til sidst sender master et stop bit, for at afslutte kommunikationen.
\end{enumerate}

% !TEX root = ../rapport.tex
\newpage
\section{Lap timer}
Formålet med Lap timeren er at have et stykke kode som måler og returnere omgangstiden, for hver omgang bilen køre. Lap timeren er baseret på "timer1" og er derfor den største timer. Lap timeren bliver udover omgangstid, også brugt af andre kode moduler, til at tage tid.
\subsection{Opsætning af Lap timer}
Som tidligere nævnt, er det "timer1" som bliver benyttet i dette kodemodul. Inden timer1 kan bruges, skal følgende siddes op:
\begin{enumerate}
\item Control register
\item Nulstille timer/counter register
\item Timer1 overflow interrupt
\item Extern interrupt 2 
\end{enumerate}



* Valg af prescaler (udregninger)\\
	-præcision
	-maks/min omgangs tider 


\subsection{Kode}
\subsection{Macro's}


	
* 24-bit register\\
% !TEX root = ../rapport.tex
\newpage

\section{Motor omdrejninger (kode)}
Dette kodemodulet er lavet til at måle tiden mellem pulsene fra motor encoderen. Ved både at kende tiden imellem pulsene og antal af pulse på en motor omgang, kan motor omdrejnings hastigheden findes.

\subsection{Kode}



\subsection{Macro's}

% !TEX root = ../rapport.tex
\newpage
\subsection{Kortlægning af bane}
Den bedste omgangstid afhænger af mange elektriske komponenter, som bindes sammen af software.
Denne software skal ud fra disse analoge og digital komponenter kunne analysere banen, og ud fra dette
kortlægge banen. Byggestenen til opbygning af denne kortlægning, kan ses på figur \ref{fig:mapflow}.

\begin{wrapfigure}{r}{0.35\textwidth}
  \vspace{-10pt}
  \begin{center}
      \includegraphics[width=0.25\textwidth]{kapitler/billeder/mapflow.png}
    \end{center}
    \vspace{-10pt}
    \caption{Viser grundstenene af koden til kortlægning af banen.}
    \label{fig:mapflow}
\end{wrapfigure}

Som der ses på figur \ref{fig:mapflow}, er pincippet bygget meget simpelt op, og kan beskrives ved følgende punkter:

\begin{itemize}
\item Venter på interrupt fra hvid-streg sensor (Tiltøj til flowchart)
\item Når den hvide streg er registeret, gemmen den omdrejningstælleren.
\item Herefter køres et loop, som bruger gyroskopet til at tjekke om der drejes.
\item Hvis vi drejer, gemmes omdrejningstæller og længden af det lige stykke regnes.
\item Status for et lige stykke, samt længden af stykket gemmes i SRAM.
\item Nyt loop tjekker om svinget er slut. Hvis sving er slut, gemmes omdrejnignstæller.
\item Status for sving, samt længde af sving gemmes i SRAM, og koden starter forfra.
\item Ved næste hvid-streg interrupt stoppes kortlægningen.
\end{itemize}

Når koden er færdig, vil alle sving og lige stykker være defineret i SRAM og koden vil blive behandlet
af et seperat program.

% !TEX root = ../rapport.tex
\newpage

\subsection{kortlæsning}

Kodemodulet "Kortlæsnings modulet" benytter et map over banen, til at køre efter. Kortlægning kan der læses mere om i afsnit \textbf{kortlægning}.\\

\begin{figure}[ht]
\centering
\includegraphics[width=0.4\textwidth]{kapitler/billeder/mread/mread_ov.png}
\caption{Overordnet flowchart over map read koden}
\label{fig:mread}
\end{figure}

Inden bilen kan køre efter kortet, er det vigtigt at bilen kender sin position på banen. Kodemodulet starter derfor med at finde målstregen, ved brug af streg sensoren. Dernæst indlæses det kommende bane segment. Hvert segment har et status register, som fortæller om det er et sving eller lige stykke. Ud over status registrer, er der også to 8-bit registre som indeholder længden af det pågældende segment. Programmet bruger informationen til at vælge imellem 2 rutiner, rutinen for sving eller rutinen for lige strækninger. Når en af de følgende rutiner er blevet udført og bilen passeret bane segmentet, loades på ny et nyt banesegment.\\
\\
Når rutinen for sving kaldes, tændes elektromagneten for at forbedre grebet. Bilens hastighed bliver sat til den maksimale hastighed for sving.\\
\\
Når rutinen for lige segment kalde, slukkes elektromagneten og hastighed for bilen siddes til max. Ud over det, tjekkes der om næste segment er lige eller et sving. Hvis næste segment er lige, bibeholdes den maksimale hastighed. Hvis næste segment derimod er et sving, beregnes hvornår nedbremsningen skal finde sted, for at nå ned på den maksimale hastighed for sving. Beregningerne for hvornår der skal bremses, sker ved ved at den ønskede bremselængden trækkes fra segmentets længde. Bremselængden er en konstant distance som kan ændres i programmet. Til nedbremsningen benyttes elektromagneten for at skabe mere friktion, samt H-broen til at vende strømmen på motoren.\\



\subsection{Delkonklusion}
Softwaren sørger for at styre bilen rundt på banen, ved at tilpasse hastigheden, til der hvor bilen befinder sig. For at kunne tilpasse hastigheden laves et kort over banen. Kortet laves ved brug af et gyroskop som kan detektere sving og en motor omdrejningstælleren som bruges til opmåling afstand. Til at håndtere dataene fra bilen, er terminalen programmeret således at der kan vises live grafer, ud fra data som bilen opsamler. Graferne kan gemmes som .csv filer, hvilket gør terminalen til et uundværligt udviklingsværktøj. Der kan konkluderes at software modulet til racerbilen virker som det skal og de stillede krav opfyldes. 