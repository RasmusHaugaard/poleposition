% !TEX root = ../rapport.tex
\subsection{Fysiske modifikationer}
\label{fysiske-modifik}

For at minimere vægten af bilen er alt overflødigt plastik blevet fjernet fra bilen.
Grunden til dette kan beskrives vha. Newtons 2. lov. Bilen bevæger sig fremad med en kraft F, og bilen har en masse m.
\begin{equation}
F=m \cdot a
\end{equation}
\begin{equation}
a = \frac{F}{m}
\end{equation}
Accelerationen a, stiger derved jo mindre massen af bilen er.

Downforce er ekstremt brugbart i sving da det hjælper med at holde bilen på banen ved højere fart. Downforce er mindre brugbart på lige strækninger, da bilen let bliver på banen og downforce vil sænke bilens acceleration og topfart. Derfor indføres der justerbar downforce med elektromagneten. I det 3D-printet chassis er der en indbygget holder til elektromagneten. Elektromagnet er placeret bagerst for at give kontra til den permamagnet der i forvejen sidder i bilen. Permamagneten er placeret foran og tæt på splitten for at trække splitten ned i banen. For uddybning af elektromagneten se sektion \ref{Elektromagnet}.

\subsubsection{3D-printet chassis}

For at kunne komme hurtigt gennem sving, kræves et lavt tyngdepunkt i bilen.
Jo lavere tyngdepunktet er, jo hurtigere kan bilen klare at køre gennem et sving uden at kæntre.
Dette fænomen kan ses på figur \ref{fig:bilsving} nedenfor.
\begin{figure}[ht]
    \centering
    \includegraphics[width=0.4\linewidth]{kapitler/billeder/bilsving.png}
    \caption{Skitse af bilen i et venstresving med to forskellige tyngdepunkter}
    \label{fig:bilsving}
\end{figure}
På bil 1 er tyngdepunktet (den blå prik) lavt og derved har den en kortere arm (den grønne streg) fra tyngdepunktet til omdrejningspunktet (den røde prik). Omdrejningsmomentet(den lilla pil) kan beskrives som:
\begin{equation}
\tau = F \cdot L
\end{equation}
Hvor $\tau$ er omdrejningsmomentet, L er armen (den grønne streg) og F er kraften der påvirker den (den sorte pil). Der antages at der skal det samme omdrejningsmoment ($\tau$) til at bilen kæntre.
\begin{equation}
F = \frac{\tau}{L}
\end{equation}
Derved kan det ses at når armen, L, bliver større skal der mindre kraft F til bilen kæntre.

Ved at fjerne affjedringen optimeres tyngdepunktet, da karrosseriet er i sin laveste tilstand hele tiden, og ikke kun når fjedrene er trykket sammen.

Det udleveret chassis på racerbilen er meget vakkelvornt, og kan vrides let.
For at optimere dette er der blevet 3D-printet et nyt forstærket chassis.
Det chassis er stærkere end det gamle og giver lettere plads til at montere sensorer og aktuatore i bilen.
Det nye chassis er tegnet i AutoDesk Inventor og derefter 3D-printet med Makerbot.
3D-printet er printet i flere dele og derefter samlet med skruer.
En fordel ved at printet består af flere dele er hvis én af delene går i stykker skal man ikke printe det hele igen.
3D-tegningen ses på figur \ref{fig:3Dtegning} nedenfor.

\begin{figure}[ht]
    \centering
    \includegraphics[width=0.45\textwidth]{kapitler/billeder/tiltview.jpg}
    \caption{3D-tegning af bilens chassis}
    \label{fig:3Dtegning}
\end{figure}


Det vil sige at den affjedring der følger med bilen er blevet fjernet og erstattet med det afstivet chassis. Grunden til man har affjedring i virkelige biler er for at beskytte både kører og motor, da man aldrig er sikker på asfaltens tilstand.
Da banen racerbilen skal køre på er flad og der ingen fører skal beskyttes, vil det optimere bilens hastighed igennem sving uden at den kæntre, at fjerne affjedringen. På figur \ref{fig:racerchassis} nedenfor ses det 3D-printet chassis med motor og hjul på.

\begin{figure}[ht]
    \centering
    \includegraphics[width=0.6\textwidth]{kapitler/billeder/racerchassis.jpg}
    \caption{3D-printet chassis med motor og hjulakser}
    \label{fig:racerchassis}
\end{figure}

Det nye 3D-printet chassis er også udviklet til at ATmega-boardet kan skrues direkte på. Det nye chassis er blevet testet mod det gamle chassis i et 180 graders sving. Med det gamle chassis var den maksimale indgangshastighed bilen kunne køre ind i et sving uden at ryge af 1.9 m/s. Med det nye chassis kan bilen kører samme sving med en indgangshastighed på 2.6 m/s uden at ryge af.
