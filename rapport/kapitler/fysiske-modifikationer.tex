% !TEX root = ../rapport.tex
\newpage
\section{Fysiske modifikationer}
For at optimere bilens omgangstider mest muligt er der også fortaget fysiske modifikationer på bilen. Som beskrevet i opgaveformulering var der også krav til at der skulle bruges enten en elektromagnet som aktuator eller en hall-sensor. Der er blevet fortrukket en elektromagnet til at skabe yderligere downforce.

De ting der blandt andet er med til at give en optimeret omganstid er:
\begin{itemize}
\item Minimering af vægt
\item Afstivelse af chassis
\item Lavere tyngdepunkt
\item Justerbar downforce
\end{itemize}
For at minimere vægten af bilen er alt unødvendigt plastik blevet fjernet fra bilen.

Det udleveret chassis på racerbilen er meget vakkelvornt, og kan vrides let. For at optimere dette er der blevet 3D-printet et nyt forstærket chassis. Det chassis er stærkere end det gamle og er samlet i et stykke.

Det vil sige at den affjedring der følger med bilen er blevet fjernet og erstartet med det stive chassis. Grunden til man har afjedring i virkelige biler er for at beskytte både kører og motor, da man aldrig er sikker på asfaltens tilstand. Da banen racerbilen skal køre på er flad og der er ingen kører der skal beskyttes, vil det optimere bilens hastighed igennem sving uden at den kæntre at fjerne affjedringen.

For at kunne hurtigt gennem sving kræver det et lavt tyngdepunkt i bilen, jo lavere tyndgdepunktet er jo hurtigere kan bilen klare at køre gennem et sving uden at kæntre. Ved at fjerne affjedringen optimeres tyngdepunktet også da karroseriet er i sin laveste tilstand hele tiden, og ikke kun når fjedrene er trykket sammen.

Downforce er ekstremt brugbart i sving da det hjælper med at holde bilen på banen ved højere fart. Downforce er mindre brugbart på lige strækninger da bilen let bliver på banen og downforce faktisk vil sænke bilens acceleration og topfart. Derfor skal der indføres justerbar downforce med elektromagneten. I det 3D-printet chassis er der en indbygget holder til elektromagneten. Elektromagnet er placeret så tæt på den split som går ned i banen som muligt for at øge downforce på den da den holder bilen fast i banen, og der sidder en permamagnet bagesrt i bilen forvejen.
