% !TEX root = ../rapport.tex

\subsection{Fysiske modifikationer}

For at minimere vægten af bilen er alt unødvendigt plastik blevet fjernet fra bilen.

Downforce er ekstremt brugbart i sving da det hjælper med at holde bilen på banen ved højere fart. Downforce er mindre brugbart på lige strækninger da bilen let bliver på banen og downforce faktisk vil sænke bilens acceleration og topfart. Derfor indføres der justerbar downforce med elektromagneten. I det 3D-printet chassis er der en indbygget holder til elektromagneten. Elektromagnet er placeret bagerst for at give kontra til den permamagnet der i forvejen sidder i bilen. Permamagneten er placeret foran og tæt på splitten for at trække splitten ned i banen. For uddybning af elektromagneten se sektion \ref{Elektromagnet}.

\subsubsection{3D-printet chassis}

For at kunne hurtigt gennem sving kræver det et lavt tyngdepunkt i bilen, jo lavere tyngdepunktet er jo hurtigere kan bilen klare at køre gennem et sving uden at kæntre. Ved at fjerne affjedringen optimeres tyngdepunktet også da karrosseriet er i sin laveste tilstand hele tiden, og ikke kun når fjedrene er trykket sammen.

Det udleveret chassis på racerbilen er meget vakkelvornt, og kan vrides let. For at optimere dette er der blevet 3D-printet et nyt forstærket chassis. Det chassis er stærkere end det gamle og giver lettere plads til at montere sensorer og aktuatore i bilen. Det nye chassis er tegnet i AutoDesk Inventor og derefter 3D-printet med Makerbot. 3D-printet er printet i flere dele og derefter samlet med skruer. En fordel ved at printet er i flere dele er hvis én del går i stykker skal man ikke printe det hele igen. 3D-tegningen ses på figur \ref{fig:3Dtegning} nedenfor.

\begin{figure}[ht]
    \centering
    \includegraphics[width=0.7\textwidth]{kapitler/billeder/3Dtegning.jpg}
    \caption{3D-tegning af bilens chassis}
    \label{fig:3Dtegning}
\end{figure}


Det vil sige at den affjedring der følger med bilen er blevet fjernet og erstattet med det afstivet chassis. Grunden til man har affjedring i virkelige biler er for at beskytte både kører og motor, da man aldrig er sikker på asfaltens tilstand. Da banen racerbilen skal køre på er flad og der er ingen kører der skal beskyttes, vil det optimere bilens hastighed igennem sving uden at den kæntre, at fjerne affjedringen. På figur \ref{fig:racerchassis} nedenfor ses det 3D-printet chassis med motor og hjul på.

\begin{figure}[ht]
    \centering
    \includegraphics[width=0.7\textwidth]{kapitler/billeder/racerchassis.jpg}
    \caption{3D-printet chassis med motor og hjulakser}
    \label{fig:racerchassis}
\end{figure}

Det nye 3D-printet chassis er også udviklet til at ATmega-boardet kan skrues direkte på. Det nye chassis er blevet testet mod det gamle chassis i et 180 graders sving. Med det gamle chassis var den maksimale indgangshastighed bilen kunne køre ind i et sving uden at ryge af 1.9 m/s. Med det nye chassis kan bilen kører samme sving med en indgangshastighed på 2.6 m/s uden at ryge af. 


