% !TEX root = ../rapport.tex
\newpage

\subsection{kortlæsning}

Kodemodulet "Kortlæsnings modulet" benytter et map over banen, til at køre efter. Kortlægning kan der læses mere om i afsnit \textbf{kortlægning}.\\

\begin{figure}[ht]
\centering
\includegraphics[width=0.4\textwidth]{kapitler/billeder/mread/mread_ov.png}
\caption{Overordnet flowchart over map read koden}
\label{fig:mread}
\end{figure}

Inden bilen kan køre efter kortet, er det vigtigt at bilen kender sin position på banen. Kodemodulet starter derfor med at finde målstregen, ved brug af streg sensoren. Dernæst indlæses det kommende bane segment. Hvert segment har et status register, som fortæller om det er et sving eller lige stykke. Ud over status registrer, er der også to 8-bit registre som indeholder længden af det pågældende segment. Programmet bruger informationen til at vælge imellem 2 rutiner, rutinen for sving eller rutinen for lige strækninger. Når en af de følgende rutiner er blevet udført og bilen passeret bane segmentet, loades på ny et nyt banesegment.\\
\\
Når rutinen for sving kaldes, tændes elektromagneten for at forbedre grebet. Bilens hastighed bliver sat til den maksimale hastighed for sving.\\
\\
Når rutinen for lige segment kalde, slukkes elektromagneten og hastighed for bilen siddes til max. Ud over det, tjekkes der om næste segment er lige eller et sving. Hvis næste segment er lige, bibeholdes den maksimale hastighed. Hvis næste segment derimod er et sving, beregnes hvornår nedbremsningen skal finde sted, for at nå ned på den maksimale hastighed for sving. Beregningerne for hvornår der skal bremses, sker ved ved at den ønskede bremselængden trækkes fra segmentets længde. Bremselængden er en konstant distance som kan ændres i programmet. Til nedbremsningen benyttes elektromagneten for at skabe mere friktion, samt H-broen til at vende strømmen på motoren.\\

