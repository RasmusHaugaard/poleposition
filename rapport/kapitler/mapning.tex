% !TEX root = ../rapport.tex
\newpage
\subsection{Kortlægning af bane}
\label{sub:mapning}
Den bedste omgangstid afhænger af mange elektriske komponenter, som bindes sammen af software.
Denne software skal ud fra disse analoge og digital komponenter kunne analysere banen, og ud fra dette
kortlægge banen. Byggestenen til opbygning af denne kortlægning, kan ses på figur \ref{fig:mapflow}.

\begin{wrapfigure}{r}{0.35\textwidth}
  \vspace{-10pt}
  \begin{center}
      \includegraphics[width=0.25\textwidth]{kapitler/billeder/mapflow.png}
    \end{center}
    \vspace{-10pt}
    \caption{Viser grundstenene af koden til kortlægning af banen.}
    \label{fig:mapflow}
\end{wrapfigure}

Som der ses på figur \ref{fig:mapflow}, er pincippet bygget meget simpelt op, og kan beskrives ved følgende punkter:

\begin{itemize}
\item Venter på interrupt fra hvid-streg sensor (Tiltøj til flowchart)
\item Når den hvide streg er registeret, gemmen den omdrejningstælleren.
\item Herefter køres et loop, som bruger gyroskopet til at tjekke om der drejes.
\item Hvis vi drejer, gemmes omdrejningstæller og længden af det lige stykke regnes.
\item Status for et lige stykke, samt længden af stykket gemmes i SRAM.
\item Nyt loop tjekker om svinget er slut. Hvis sving er slut, gemmes omdrejningstæller.
\item Status for sving, samt længde af sving gemmes i sram, og koden starter forfra.
\item Ved næste hvid-streg interrupt stoppes kortlægningen.
\end{itemize}

Når koden er færdig, vil alle sving og lige stykker være defineret i SRAM og koden vil blive behandlet
af et separat program.
