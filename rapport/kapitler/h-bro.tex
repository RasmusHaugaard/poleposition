% !TEX root = ../rapport.tex

\subsection{Bremsefunktion (H-bro)}

Ligesom acceleration, er bremselængden en vigtig faktor for at opnå en god omgangstid.
Når bilen nærmer sig et sving, handler det om at komme så tæt på svinget som muligt, før at bilen
skal begynde at bremse. Dette kræver en effektiv bremse, så bremselængden minimeres og omgangstiden effektiviseres.
For at bremse bilen så effektivt som muligt, benyttes princippet bag en H-bro. Dette princip er illustreret
på figur \ref{fig:hbro}, nedenfor.

\begin{figure}[ht]
    \centering
    \includegraphics[width=1\textwidth]{kapitler/billeder/hbro.png}
    \caption{Viser princippet bag en H-bro, og hvordan den benyttes i dette projekt.}
    \label{fig:hbro}
\end{figure}

Som illustreret på figur \ref{fig:hbro}, kan man bruge en H-bro til at vende strømmen igennem motoren.
Ved at vende strømmen igennem motoren i få millisekunder, vil motoren prøve at bakke,
og bilen vil bremse meget effektivt. Figuren viser det to scenarier, hvor strømmen til højre er vendt
og strømmen til venstre ikke er vendt.

H-broen konstrueres med et dobbelt relæ, som vender de to kontakter når spolen inducere en strøm.
Spolen kræver en høj aktiveres spænding for at trække den nødvendige strøm
og kredsløbet bygges derfor op med en transistor.
Transistoren kan aktiveres med en meget lille strøm og spænding, og kan derfor aktiveres
af ATmega'ens I/O pins.

Bilens motor kan derved bremses, ved at sætte en I/O pin høj, og derved vende strømmen igennem motoren.

(Eventuelt beskrivelse af komponenterne vi har brugt + at nævne det kan skade gearingen.)
