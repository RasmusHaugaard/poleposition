% !TEX root = ../rapport.tex
\section{Udviklingsmiljø}

En stor del af opgaven ligger i at programmere microkontrolleren. Der er derfor lagt en del energi i at optimere gruppens udviklingsmiljø.

Fem personer har skulle skrive moduler i assembly-kode, der skulle virke med andre moduler, samt testprogrammer, der inkluderede disse moduler. Disse testprogrammer skulle flashes ned på microkontrolleren.

\subsection{Fildeling}
Det var nødvendigt, gruppens medlemmer kunne dele kode med hinanden. Dropbox og Github blev overvejet.
Dropbox forsøger at holde gruppens medlemmers filer synkroniseret, hurtigst muligt.
Det gør det nemt og hurtigt at arbejde med.
Github (Git) tilbyder et online arkiv med versionsstyring, man aktivt skal hente fra og uploade til.

Grundet et ønske om, man kunne isolere sig fra ændringer i perioder og teste egen kode uden risiko for, ændringer fra andre medlemmer medfulgte fejl, samt sikkerheden i versionsstyringen, blev Github valgt til at hoste al kode samt rapporten.
Dropbox blev valgt til at hoste diverse filer som datablade, test videoer, billeder, 3D filer med mere.

\subsection{Filedef}
Assembleren avra har direktivet ``.def'' til at tildele registre navne. For eksempel, ``.def temp = R16''.
For at kunne skrive moduler, der var så uafhængige af hinanden som muligt og kunne tildele registre navne begrænset af konteksten af modulet, blev der skrevet en preprocessor med et nyt direktiv, ``.filedef''. For eksempel, ``.filedef temp = R16''. Til forskel for ``.def'' vil ``.filedef'' som navnet antyder kun gælde i den fil, definitionen eksisterer. Preprocessoren er skrevet i nodejs og køres før assembleren i build setuppet.

\subsection{Build setup}
For at kunne builde på både osx, linux og windows er build setuppet også skrevet i nodejs.

% !TEX root = ../rapport.tex
\section{Bootloader}
Programmerstikket til mikroprocessoren er ikke nem at tilgå, når bilen er fuldt monteret. For at kunne iterere hurtigt og undgå at skulle skille bilen ad og samle bilen hele tiden, blev der udviklet en bootloader.

\subsection{Bluetooth}
For at sikre, der kan skabes kontakt med bootloaderen, styrer bootloaderen bluetoothkommunikationen.
\subsubsection{Baudrate}
En række stresstests blev udført for at finde den hurtigste, stabile baudrate mellem mikrocontrollerens og bluetoothmodulet. Dataflowet er primært fra mikrocontrollerens til pc. Stresstestene er opbygget derefter.
En enkelt byte med værdien, x, sendes til mikrocontrollerens, hvorefter mikrocontrollerens sender x kb tilbage.
Se ``avr/src/bt/tests/replykb.asm''.
Forskellige baudrates blev stresstestet ved 250 kB. Den højeste af disse, der ikke havde datatab, var 38 kbps. Disse tests blev udført uden parity bit, med én stop bit og med 8 databits. Det er også denne opsætning der antages fremadrettet. Se eventuelt ``avr/src/bt/bt\_setup.asm''. Det vil sige 9 bits i alt per overført byte. Det giver en teoretisk overførselshastighed på $38 kbps / 9 bpB \simeq 4.2 kB/s$ Der kunne have været eksperimenteret med parity bits og flere stopbits ved højere baudrates, men der var ikke umiddelbart behov for hurtigere overførselshastigheder.
\subsubsection{Output Buffer}
Med en clockfrekvens på 16 MHz går der $16*10^6 / 4.2*10^3 \simeq 3810$ cycles mellem, der kan sende bytes. Hvis man ønsker at sende $x$ bytes synkront, skal programmet i alt vente $(x-1)*3810$ cycles. For at udnytte disse cycles i stedet for at vente, implementeres en buffer i sram.
Se ``avr/src/bt/bt\_tr.asm''.
Bufferen består af en byte liste med en fast længde allokeret i sram samt to pointers, der også er allokeret i sram. Listen indeholder de bytes, der ønskes sendt, samt den frie plads i bufferen. Bytes i det frie område kan være bytes, der allerede er sendt, eller bytes, der endnu ikke er blevet skrevet over siden mikrocontrollerens sidste reset. En store pointer peger på den byte i listen, der blev tilføjet sidst. En send pointer peger på den byte i listen, der blev sendt sidst.
I initialiseringen af bufferen sættes de to pointere til at pege på den samme byte i listen.
Når der ønskes en byte sendt, inkrementeres store pointeren, og byten skrives til den adresse i listen store pointeren repræsenterer. Interrupt enableren for når UART data registeret er tomt sættes højt, så bufferen bliver informeret, når der kan sendes en byte.
Når der kan sendes en byte, og der er bytes i bufferen, der endnu ikke er sendt (store pointeren og send pointeren har ikke samme værdi), inkrementeres send pointeren, og den byte, send pointeren peger på sendes.

Hvis en af pointerne, efter at blive inkrementeret, peger på byten lige efter listens sidste byte, sættes pointeren til den første byte i listen.
Hvis store pointeren, efter at blive inkrementeret og eventuelt rykket til den første byte i listen, peger på samme adresse som send pointeren, er der overflow i bufferen. Det håndteres på nuværende tidspunkt ikke.
TODO: Der skal et billede ind her

\subsection{Protokol}
Bootloaderen styrer bluetoothkommunikationen og indeholder en overordnet kommunikationsprotokol.
Den første byte, initial byte, bootloaderen modtager, afgør, hvordan bootloaderen skal håndtere de næstkommende bytes.
85: Set. 85 gemmes som den første byte i inputbufferen. De to næstkommende bytes gemmes også, før der hoppes til kommando interruptet. Herefter forventer bootloaderen en ny initial byte.
87: Ping. Bootloaderen sender 87 tilbage med det samme. Herefter forventer bootloaderen en ny initial byte.

For at undgå at skulle opdatere bootloaderen, hvis der blev implementeret andre kommandoer i applikationsprotokollen, blev der implementeret en variabel kommando med initial byte: 86.
Den næstkommende byte beskriver længden, n, af kommandoen. De derefter næstkommende, n, bytes gemmes i inputbufferen, før der hoppes til kommando interruptet.
Bemærk:
[85, X, X] og [86, 3, 85, X, X] er ekvivalente.
[87] er ping kommandoen til bootloaderen. [86, 1, 87] er en kommando [87] til applikationen.

Længden er kun variabel for bootloaderen. Der gives ingen information til applikationen om længden af kommandoen.

Når der er en hel kommando (liste af bytes) klar til applikationen, hoppes til 0x2A, en udvidelse af interrupt vektor tabellen.

